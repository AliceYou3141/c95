\section*{謝辞}
\begin{center}
この本を読んでくださる方に \\
気力をくれる友人に \\
大切な人に \\
感謝と本書をささげます
\end{center}

\section*{前書き}
LDAPのディレクトリをデザインしようとしたとき、資料がないことに気づきます。
Postfixだけでなく、LDAPだけでもありません。インフラに関する本が、少なくなっています。

Postfixでメールシステムを構築するときに、ユーザアカウントデータベースとして、LDAPやRDBMSを使用することができます。本書は、LDAPを、メールシステムのユーザデータベースとして使用する、という観点に限って解説したものです。

LDAPは機構増のメタファーを持ったデータベースです。そのため、RDBMSとはまた別のとっつきにくさがあります。本書では、最終的な利用目的を絞った形ではありますが、LDAPそのものについての理解の一助になれば、幸いです。





\section*{本書の内容}
本書は、メールアカウントデータベースとしてのLDAPについてまとめた本です。PostfixでLDAPを利用するために必要な知識をまとめたものなので、属性の追加の詳細やPENの取得など、内容を省いた部分もあります。

\paragraph{第一章}
ディレクトリという概念と、それをサービスとして実装したLDAPについて説明します。LDAPはディレクトリの概念をどのようにサービス化しているかを解説しています。

\paragraph{第二章}
LDAPでユーザデータベースを作ることの概要について説明しています。その説明を通じて、LDAPはどのようにデータを持つかと、アクセスコントロールについても説明します。

\paragraph{第三章}
LDAPをメールアカウントデータベースに使用するための設計について説明します。特に、Postfixと組み合わせて使う前提で、Postfixテーブルをどのように配置するかについても、解説しています。

\paragraph{第四章}
PostfixからLDAPを、メールアカウントとPostfixテーブルのデータベースとして利用する方法を説明します。
Postfixから検索を行うための探査構文の書き方と、Postfixテーブルの置き換え例についても説明しています。


\section*{免責事項}
本書に書いてあることは、筆者知識のレベルでまとめたものです。ですが、内容が正しいとは言い切れません。初版でも改訂版でも相当やらかしています。また、学校のレポート、業務などのコードを書く際に、本書の内容を信じて書いて損害が生じても、筆者にその責任はありません。

くれぐれも、自己責任と十分な検証の上、ご利用ください。

\section*{表紙イラスト}
ゆうちゃん (コース英知)