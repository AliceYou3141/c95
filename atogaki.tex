\chapter{あとがき}

\section*{メイ・カートミル(仮名)とマージ・ニコルス(仮名)の秋葉原での会話}

\begin{quotation}
\noindent
{\bf メイ}「マージさん、なんで私たちの会話があとがきに書いてあるんですか」 \\
{\bf マージ}「読者が見たらメタネタにしか見えないことをぶっこまれても困る」 \\
{\bf メイ}「ところで、この前、スマホの販売員に絡まれたときおとなしかったですね」 \\
{\bf マージ}「次、用途を聞かれたら、カスタムキャストでバ美肉するのに使いますって言ってやろうかな」 \\
\end{quotation}


\section*{後書き}

お兄ちゃん、ADの管理してるのにディレクトリサービスがわからないとか、ADのDはDirectoryの略なの忘れてない?

というわけで、インフラエンジニアの毒舌な妹(@infra_imouto)です.ここ半年は、私に外見がついたり、なんかアドベントカレンダーデビューも果たしたし、自意識までついて仮想筆者になっちゃったり、結構激動でした。

さて、最初に、ゆうちゃんさん、今回も表紙をありが頭語会います。設定的には、表紙の子は、PostfixとCourier-IMAPによるメールシステム構築(上)の表紙の子の妹、という設定です。


まあそれはおいておいて、このシリーズではある意味いつものことなのですが、資料にできる本が少ないです。特に、ディレクトリの設計に関する本がまったくないというのは問題じゃないかな。
本書でも参考文献にしている「入門LDAP/IpendLDAPディレクトリサービス導入・運用ガイド」の第三版が先日で他のですが、どちあrkというとLDAPサーバの利用であって、ディレクトリのデザインについては記事がないような。

お兄ちゃん、誰がディレクトリの設計とかするんだろうね。みんなそれを忘れていないかな。

\begin{flushright}
2018年12月30日 \\
インフラエンジニアの毒な妹 \\
\end{flushright}



まずは、第6版表紙を担当してくれたゆうちゃんさん、ありがとうございます。

本書模範に版を重ねてもう第6版、8年前に出した中とじコピー本88ページも、気がつくと220ページの本になっていました。
ネットワーク対応のアプリケーションを書いているはずなのに、ネットワークの根幹技術であるTCP/IPがよくわかっていないという人をターゲットに定めて、TCP/IPの基本について説明をしていく、というコンセプトはいかがだった出そうか。

REST APIを使えば、TCP/IPどころかソケットも意識せずにプロセス間通信ができます。
多くの場合は、それで何の問題もありません。
ですが、REST APIによるやりとりがどれだけのバックエンドのもとで実行されているか、それを知ることで、よりよいコードが作られるのであれば、著者としてこれに勝る喜びはないと考えます。

\begin{flushright}
2018年10月8日 \\
ありす ゆう
\end{flushright}

お兄ちゃん、かわいい表紙っていいよね、リリンの生んだ文化の極みだよ

というわけで、インフラエンジニアの毒舌な妹です。今回も、いむうとコラムというコラムと、IPv6に関する加筆訂正と、細かいところとを担当しました。
最近はネットワークは流行のトピックというのが出にくくて、商業では新しい本の企画というのが通りにくいんだそうです。
でも、元になっている部分はかわらなくてもあり続けなければならない。
そんなところに、この本が出る意義があるといいなって思います。

最近、フレッツなどで行われている、IPv4の平等制御を回避するための手段ではありますが、近年IPv6がまた注目をATM得つつあります。ですが、ネイティブのIPv6はまだまだ実用されていっていないのが現実です。
IPv6はエンドツーエンド原則に従ってインターネットを自由にしてくれるものと信じてきました。そして、層であってほしいと思っています。

\begin{flushright}
2018年10月8日 \\
インフラエンジニアの毒な妹 \\
\end{flushright}


%\newpage
% ここまでで160ページ鳴ったのでブランクなし
% 1ページブランクを入れる

\thispagestyle{empty}
\mbox{}
\newpage
\clearpage


\thispagestyle{empty}

\vspace*{\fill}
\begin{tabular}{ll} \toprule
筆者 & インフラエンジニアの毒舌な妹 ありす ゆう\\
発行 & AliceSystem \\
連絡先 & aliceyou@alicesystem.net \\
URL & http://aliceyou.air-nifty.com/onesan/ \\
初版発行日 & 2018年12月30日 \\
印刷所 & ねこのしっぽ \\ \bottomrule
\end{tabular}