\chapter{LDAPによるユーザデータベース}

\section{なぜLDAPをユーザデータベースに使うのか}

ユーザアカウント管理を外部化するとき、RDBMSやNoSQLでなく、LDAPを使用することが多いです。MicroSoftのActive Directoryも、リソース管理にLDAPを使用しています。
では、LDAPをユーザデータベースに使う理由は何でしょうか。

\subsection{LDAPと組織図の親和性}
一般に、組織図は木構造を取ります。その場合、ユーザは、木構造のリーフノードとして置かれます。このような構造をそのまま表すには、木構造データベースであるLDAPが適しています。
たとえば、スクールアイドル部という部署で、ユニットが三つあります。各ユニットには、3人ずつメンバーがいます。
この場合、スクールアイドル部をルートノードにした、木構造であらわすことができます。

この程度なら、RDBMSでも、一対多の関連があるテーブルの連鎖で表すことができるでしょう。
ですが、ユーザアカウントを検索するとき、どこのユニットに属しているかという情報なしで、探す必要があります。そのようなときに、LDAPであれば、スクールアイドル部以下の子ノードを、再帰的に探索するだけです。
一方、SQLのSELECT文で探索を行うのは面倒です。三つのユニットの空間をUNIONして探すとしても、各ユニットの属性が揃っている必要があります。

\subsection{構造の自由度}

LDAPは、ある親ノードの下にある子ノードに、何を置いてもかまいません。先ほどの例では、ユニットの下にメンバーのノードを置いていました。このノードと一緒に、ユニットの代表メールアドレスの情報をDNとして持つノードを置くこともできます。

それだけだと性質の違うノードをおけるだけのように思えます。ですが、LDAPの場合は、属性が同じであればこの両者を一度に検索対象とすることができます。

たとえば、ユニットのメンバーのノードと、代表メールアドレスのノードが、それぞれ同じ属性mailを持っているとします。このとき、あるメールアドレスがユニットのメンバーもしくは代表メールアドレスであるかを調べたい場合、ユニットの子ノードすべてを検索対象として、mail属性に該当の者があるかを検索すればよいことになります。


\subsection{LDAP探索の計算量}

LDAPは、DNによってあるノードを指定し、そのノードの子ノードを検索対象とする、もしくは、その子ノード以下を再帰探索の対象とするという探し方をします。
このとき、指定したノードの子ノードを探索対象とする場合の計算量がいつも同じという特徴があります。共通の親を持つ兄弟ノードの中から検索するとき、計算量がいつも同じです。

一方、その兄弟ノードがそれぞれ子となるノードを持っており、そのノードも探索の範囲であった場合、再帰探索がおこなわれます。そして、再帰の深さが分からないため、探索かかる時間を予測することができなくなります。

\subsection{書き換え頻度}

LDAPは、頻繁な書き換えには向かない、という欠点があります。木構造であるということは、その構造そのものを書き換えるにはコストがかかります。また、データもハッシュテーブルであるBerkeleyDBをバックエンドにしている実装もあり、書き換えのパフォーマンスはあまり高くありません。

ですが、ユーザアカウントデータベースは、書き換え頻度が少ないという特徴があります。そのため、RDBMSほどの書き換えパフォーマンスは必要なく、LDAPを使う上での問題にはなりません。

\section{LDAPのノード}

LDAPのノードは、一つ以上の属性を含みます。その属性の取捨選択は、どのように行うのでしょうか。

スキーマという属性のセットをノードごとに選択することで

\subsection{属性と属性値}

属性は、ノードが持つ一つ一つの情報です。その属性に対応した値が、属性値となります。
ひとつの属性は、その値としての属性値をひとつ取ります。例えば、mailという属性は、メールアドレスをひとつその属性値として取ります。複数のメールアドレスを、その属性値としてることはありません。

属性は、ひとつのノードにひとつしか置けないものと、複数置くことができるものとがあります。これは、後述する属性の定義によります。たとえば、ユーザアカウント情報のノードでは、ユーザ名はひとつしか採ることができないように定義します。その一方、メールアドレスは、複数採ることができるように定義します。

もし、ひとつのノードに情報としての属性値を複数持たせたいときは、同じ属性を複数持たせます。たとえば、あるユーザアカウントが、メール8アドレスをtsushima.yoshiko@uranohosi.exampleとyohane@uranohoshi.exampleの二つ持っていれば、mailto属性を二つ持つノードとして設定します。そうすることで、複数のメールアドレスをひとつのユーザアカウントの情報として持たせます。



\subsection{オブジェクトクラス}

あるノードがどのような属性を含むかは、オブジェクトクラスという属性セットを選択することで決まります。この属性セットは、ひとつ、もしくは複数設定することができます。また、オブジェクトクラスはノードごとに選択します。

あるノードが使用するオブジェクトクラスは、属性objectClassの属性値として定義されます。objectClassは、一つのノードで複数現れてもかまわない属性です。

代表的なオブジェクトクラスとして、inetOrgPersonがあります。このスキーマは、ユーザアカウント情報に必要な、属性値のセットを含んでいます。
また、ミドルウェアやアプリケーションで、専用の属性を追加するためのオブジェクトクラスを追加するためのファイルが添付されていることがあります。
例として、Courier-IMAPは、専用のオブジェクトクラスの定義ファイルを、ソースツリーに添付しています。


\subsection{スキーマ}
LDAPでスキーマと呼ばれるのは、オブジェクトクラスの定義です。スキーマを読み込ませることで、そのDITで使用可能なオブジェクトクラスが定義されます。
この定義はテキストで記述され、、スキーマファイルと呼ばれます。このスキーマファイルは、以下のように記述されています。

\begin{verbatim}
attributeType( 2.5.4.41 NAME 'name'
DESC 'name(s) associated with the object'
EQUALITY caseIgnoreMatch
SUBSTR caseIgnoreSubstringsMatch
SYNTAX 1.3.6.1.4.1.1466.115.121.1.15{32768} )
attributeType( 2.5.4.3 NAME ( 'cn' 'commonName' )
DESC 'common name(s) assciatedwith the object'
SUP name )
\end{verbatim}

スキーマファイルは、LDAPサーバに添付されているものと、ミドルウェアなどが独自に使用するために添付されているものとがあります。


\subsection{OID}

スキーマファイルの中で、2.5.4.41や、1.3.6.1.4.1.1466.115.121.1.15という数字の列が現れています。これは、OID(Object ID)と呼ばれる、オブジェクトを一意的に識別するためのものです。ISO/ITU-Tで管理されています。

これは、LDAPが元とも、ITU-Tで定義されたディレクトリサービスを簡易化した実装であることに撚ります。いわば、昔のディレクトリ定義の名残を引きずっている部分です。

SNMPで使うときなど、OIDはデバイスを抽象的に表現するためにも使われます。この場合は、MIB(Management Information Base)と呼びます。

2.5.4というのはディレクトリサービスの属性であることを著わすために定義されているOIDです。その後の数字が、何を表す属性だえるかを著わすIDとなっています。また、1.3.6.1.4.1というのは、この次に、プライベートな組織を著わすIDが置かれます。そして、その先のIDは、その組織で自由に定義してよいことになっています。
この組織のIDを、PEN(Private Enterprise Number)と呼びます。PENが1466は、これ以降でデータの形式を定義するのに使われるOIDです。

1.3.6.1.4.1.(PEN)というOIDを、プライベートMIBと呼びます。これは主にSNMPでの呼び方でアリ、プライベートに定義され、他のOIDとカブラ解雇とが保証されたたOIDであると考えて構いません。また、OIDのかぶりを避けるため、独自にOIDを定義するときは、PENを取得する必要があります。
PENは、IANAに申請すれば無料で取得可能です。この取得方法は、付録に記載します。

独自のスキーマを作成するときは、OIDとして1.3.6.1.4.1.(PEN)ではじまるOIDを割り当てるようにします。





\section{アクセスコントロール}

LDAPのアクセスコントロールは、どのような概念なのでしょうか。それは、部分空間内のノードに対して、どのような捜査を許可するか、という権限設定をします。

\subsection{アクセス権とスコープ}

LDAPのアクセスコントロールは、あるノード基準にして、その子ノード以下に再帰的に適用します。つまり、LDAPのアクセス権のスコープは、部分空間がその単位になります。

たとえば、スクールアイドル部という部署の下にユニットが3つある場合で、標準的に行われる設定を考えてみましょう。

スクールアイドル部のノードを親として、各ユニットは子ノードとなります。また、ユニットの子ノードとして、3人のメンバーのリーフノード、代表メールアドレスのリーフノードがあるとします。

各メンバーは、自分のアカウントのノードには読み書きアクセスできるとします。ほかのメンバーのリーフノードや、メールアドレスのノード、親ノードにはアクセスできません。これは、パスワードを自分で変更したりするのに必要なACL設定です。

ユーザ認証ノデータベースとしてLDAPを使うときは、LDAPにアクセスするための認証なしでアクセスしたいことがあります。たとえば、メンバーのリーフノードが、何らかのサービスにログインするときの認証情報を含んでいる場合です。
このときは、LDAPアクセス認証なしの場合は、ほかのサービスが認証情報の判定をリクエストした場合その結果を返す場合は許可、として用に設定します。

ユニットの管理者は、ユニットのメンバー、メールアドレスについて読み書きできるようにします。これは、ユニットの子ノードに対して、読み書きのACLを持っていると言うことです。それでも、ユニットを表すノードや、ほかのユニットのノードにはアクセスできないようにします。

LDAPそのものの管理者のアカウントでは、ルートノード以下すべてのノードの読み書きができるようにします。

\subsection{LDAPのアクセス権}

LDAPへのアクセス権には、どのようなものがあるのでしょうか。主なものを、OpenLDAPでの設定値を併記して節瞑していきます。

\paragraph{none}
アクセス許可しないということです。部分区間に対してアクセスさせない設定を行うときに

