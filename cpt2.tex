\chapter{LDAPによるユーザデータベース}

\section{なぜLDAPをユーザデータベースに使うのか}

ユーザアカウント管理を外部化するとき、RDBMSやNoSQLでなく、LDAPを使用することが多いです。MicroSoftのActive Directoryも、リソース管理にLDAPを使用しています。
では、LDAPをユーザデータベースに使う理由は何でしょうか。

\subsection{LDAPと組織図の親和性}
一般に、組織図は木構造を取ります。その場合、ユーザは、木構造のリーフノードとして置かれます。このような構造をそのまま表すには、木構造データベースであるLDAPが適しています。
たとえば、スクールアイドル部という部署で、ユニットが三つあります。各ユニットには、3人ずつメンバーがいます。
この場合、スクールアイドル部をルートノードにした、木構造であらわすことができます。

この程度なら、RDBMSでも、一対多の関連があるテーブルの連鎖で表すことができるでしょう。
ですが、ユーザアカウントを検索するとき、どこのユニットに属しているかという情報なしで、探す必要があります。そのようなときに、LDAPであれば、スクールアイドル部以下の子ノードを、再帰的に探索するだけです。
一方、SQLのSELECT文で探索を行うのは面倒です。三つのユニットの空間をUNIONして探すとしても、各ユニットの属性が揃っている必要があります。

\subsection{構造の自由度}

LDAPは、ある親ノードの下にある子ノードに、何を置いてもかまいません。先ほどの例では、ユニットの下にメンバーのノードを置いていました。このノードと一緒に、ユニットの代表メールアドレスの情報をDNとして持つノードを置くこともできます。

それだけだと性質の違うノードをおけるだけのように思えます。ですが、LDAPの場合は、属性が同じであればこの両者を一度に検索対象とすることができます。

たとえば、ユニットのメンバーのノードと、代表メールアドレスのノードが、それぞれ同じ属性mailを持っているとします。このとき、あるメールアドレスがユニットのメンバーもしくは代表メールアドレスであるかを調べたい場合、ユニットの子ノードすべてを検索対象として、mail属性に該当の者があるかを検索すればよいことになります。


\subsection{LDAP探索の計算量}

LDAPは、DNによってあるノードを指定し、そのノードの子ノードを検索対象とする、もしくは、その子ノード以下を再帰探索の対象とするという探し方をします。
このとき、指定したノードの子ノードを探索対象とする場合の計算量がいつも同じという特徴があります。共通の親を持つ兄弟ノードの中から検索するとき、計算量がいつも同じです。

\subsection{書き換え頻度}

LDAPは、頻繁な書き換えには向かない、という欠点があります。木構造であるということは、その構造そのものを書き換えるにはコストがかかります。また、データもハッシュテーブルであるBerkeleyDBをバックエンドにしている実装もあり、書き換えのパフォーマンスはあまり高くありません。

ですが、ユーザアカウントデータベースは、書き換え頻度が少ないという特徴があります。そのため、RDBMSほどの書き換えパフォーマンスは必要なく、LDAPを使う上での問題にはなりません。

\section{LDAPのノード}

LDAPのノードは、一つ以上の属性を含みます。その属性の取捨選択は、どのように行うのでしょうか。

スキーマという属性のセットをノードごとに選択することで

\subsection{属性と属性値}

属性は、ノードが持つ一つ一つの情報です。その属性に対応した値が、属性値となります。
属性と属性値は一対一対応です。例えば、mail



\subsection{オブジェクトクラス}

あるノードがどのような属性を含むかは、オブジェクトクラスという属性セットを選択することで決まります。この属性セットは、ひとつ、もしくは複数設定することができます。また、オブジェクトクラスはノードごとに選択します。

オブジェクトクラスは、属性objectClassの属性値として定義されます。objectClassは、一つのノードで複数現れてもかまわない属性です。

代表的なオブジェクトクラスとして、inetOrgPersonがあります。このスキーマは、ユーザアカウント情報に必要な、属性値のセットを含んでいます。
また、ミドルウェアやアプリケーションで、専用の属性を追加するためのオブジェクトクラスを追加するためのファイルが添付されていることがあります。
例として、Courier-IMAPは、専用のオブジェクトクラスの定義ファイルを、ソースツリーに添付しています。


\subsection{スキーマ}
LDAPでスキーマと呼ばれるのは、オブジェクトクラスの定義です。スキーマを読み込ませることで、そのDITで使用可能なオブジェクトクラスが定義されます。
この定義はテキストで、スキーマファイルとyばれます。

スキーマファイルは、LDAPサーバに添付されているものと、ミドルウェアなどが独自に使用するために添付されているものとがあります。

\section{アクセスコントロール}

LDAPのアクセスコントロールは、どのような概念なのでしょうか。それは、部分空間内のノードに対して、どのような捜査を許可するか、という権限設定をします。

\subsection{アクセス権とスコープ}

LDAPのアクセスコントロールは、あるノード基準にして、その子ノード以下に再帰的に適用します。つまり、LDAPのアクセス権のスコープは、部分空間がその単位になります。

たとえば、スクールアイドル部という部署の下にユニットが3つある場合で、標準的に行われる設定を考えてみましょう。

スクールアイドル部のノードを親として、各ユニットは子ノードとなります。また、ユニットの子ノードとして、3人のメンバーのリーフノード、代表メールアドレスのリーフノードがあるとします。

各メンバーは、自分のアカウントのノードには読み書きアクセスできるとします。ほかのメンバーのリーフノードや、メールアドレスのノード、親ノードにはアクセスできません。これは、パスワードを自分で変更したりするのに必要なACL設定です。

ユーザ認証ノデータベースとしてLDAPを使うときは、LDAPにアクセスするための認証なしでアクセスしたいことがあります。たとえば、メンバーのリーフノードが、何らかのサービスにログインするときの認証情報を含んでいる場合です。
このときは、LDAPアクセス認証なしの場合は、ほかのサービスが認証情報の判定をリクエストした場合その結果を返す場合は許可、として用に設定します。

ユニットの管理者は、ユニットのメンバー、メールアドレスについて読み書きできるようにします。これは、ユニットの子ノードに対して、読み書きのACLを持っていると言うことです。それでも、ユニットを表すノードや、ほかのユニットのノードにはアクセスできないようにします。

LDAPそのものの管理者のアカウントでは、ルートノード以下すべてのノードの読み書きができるようにします。

\subsection{LDAPのアクセス権}

LDAPへのアクセス権には、どのようなものがあるのでしょうか。主なものを、OpenLDAPでの設定値を併記して節瞑していきます。

\paragraph{none}
アクセス許可しないということです。部分区間に対してアクセスさせない設定を行うときに

\section{メタディレクトリ}

\subsection{メタディレクトリによるマルチドメイン構成}