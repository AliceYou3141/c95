\chapter{ツリー構造の設計}

\subsection{大まかな形}

メールアカウントとして使うLDAPツリーを構成するとき、どのようなツリー構造にすれば良いのでしょうか。
慣例的に、LDAPツリーのルートノード以下は、LDAPそのものの管理アカウント情報を入れる部分木と、LDAPで管理したい情報を入れる部分木から構成されます。
今回、LDAPで管理したい情報はメールに関するアカウント情報です。

また、aliases(5)など、左辺と右辺がい一対一であるPostfixテーブルで管理するデータはLDAPで管理することが可能です。
transport(5)のように、Postfixテーブルで左辺にワイルドカードを伴うなど、右辺と左辺が一対一でないテーブルは、LDAPに入れることはできません。この点は注意をしてください。

LDAPで管理したい情報の中で、メールアカウントに関する部分木と、鋭意リアスに関する部分木を分割します。これが、今回構築するメールアカウントデータベースの大まかな形です。

\subsection{ルート部分}

ルート部分は、慣例として、メールアドレスを管理するドメインのTLDをDNとします。たとえば、uranohoshi.exampleであれば、DN:dc=exampleというようにします。dcは、domain contextという属性です。慣例的に、この下にdcでセカンドレベル以降のドメインをつなげます。これは、dcという属性の標準的な使い方です。

複数のドメインに着いて管理したい場合は、ルートをたとえば、dc:rootというようにルートであるように記述して、その子ノードとして、dc=uranohoshi.exampleと、dc=otonoki.exampleを置き、それを起点とする部分木をそれぞれのドメインのデータベースにする、という方法もあります。

ですが、コンテナなどでPaaSインスタンスを調達しやすい現在では、ドメイン毎に独立したLDAPインスタンスを用意する方がよいという考えもあります。この方法を採る場合、ACLの構成をシンプルにできるというメリットがあります。欠点は、ドメイン毎にLDAPサーバがひとつ立ち上がっている状態になる、とういうことですが、リソース調達の容易な現代では甘利問題にならないでしょう。

\subsection{ドメイン名以下の子ノード}

ルートノード以下は、管理アカウント、ユーザアカウント、エイリアスなどPostfixテーブルのサブツリーに分けます。ここでは、ユーザアカウントと、Postfixテーブル部分について解説します。

それぞれのRDNを著わすのに使う属性は、オブジェクトタイプtopに含まれる属性のouを使います。本来は、Organization Unitで、組織内の部門を著わすのに使う属性値ですが、LDAPツリーの部分木の起点である子とを著わすために、便宜的に使用されます。

ここでは、ou=usersとou=tablesというRDNをもつ二つのノードを作ることにしましょう。

\subsection{ユーザアカウント情報}

ユーザアカウント情報の部分木は、ユーザ情報とメールアドレスの関係を参照するのに使う部分木です。この部分は、ひとつひとつのユーザアカウントに対応するノードで構成されます。この部分木をどう作るか、ノードにどんな除法を含めるかについて、説明します。

\subsection{ユーザアカウントの部分木の構造}

ユーザアカウント情報は、ou=usersのノード以下に、全てのメールアカウントがある構造にします。こうすることで、ユーザアカウントに関する検索を、高速化することができます。
ユーザのグループ分けが必要な場合は、ユーザアカウントのノードにgroup属性やgid属性を置き、その値でグループ分けを行うようにします。

ou=usersのしたに、ユーザグループとしての部分木の起点として、group=グループ名という子ノードを置き、その子ノードとして、グループに所属するユーザからなる部分木を作るやり方もあります。ですが、この方法は、グループ単位でユーザ管理のACLを設定したい場合でないなら、推奨しません。
これは、ユーザやメールアドレスをキーに検索するときに、条件としてグループが判明していない場合が多々あります。この場合、部分木の再帰検索となるため、時間が多くかかります。
また、人事など、アナログな組織に対応してグループ分けをした場合など、そのユーザアカウントに対応するユーザの異動などで、別のグループに移ったとき、そのアカウントをDITの別の部分に移さなければならない、という問題が生じます。

\subsection{ユーザアカウントのノード}

ユーザアカウントに関するノードは、どのような情報を入れれば良いのでしょうか。基本的に、オブジェクトタイプinetOrgPersonをもとにしておけば多くの場合用が足ります。
inetOrgPersonは、認証を伴うユーザアカウントを定義するオブジェクトタイプです。DNには、ユーザIDである属性usaerNameを使います。

例えば、matsuura.kanan@

\subsection{実ユーザとバーチャルユーザ}





\subsection{Postfixテーブル情報}

Postfixテーブルで、左辺がひとつに定まるものは、LDAPのノードに入れることができます。これは、左辺がワイルドカードや正規表現などで、複数の値がマッチングするものは、LDAPには入れられないということです。ただし、左辺で取りうる値を全て網羅し、一つ一つの子ノードに展開してやることで、そのPsotfixテーブルをLDAPに収容することが可能です。

わかりやすくするために、`ou=tablesの子ノードとして、ou=aliases、ou=canonical、というように、どのテーブルに対応する部分木かがわかるようにノードを追加します。

例として、左辺がひとつに定まるものとして、エイリアステーブルを考えてみましょう。エイリアステーブルは、左辺がひとつのメールアドレス、右辺が一つ以上のメールアドレスになります。
以下のようなエイリアステーブルを、LDAPののーどにして、LDIFにしてみましょう。

\begin{verbatim}
# lily@uranohoshi.exampleのエイリアステーブル
lily:  sakurauchi.riko@uranohosi.example , 
       sakurauchi@otonoki.example
\end{verbatim}

この子ノードは、DNとして、mail=likyを取るようにします。また、転送先を洗わず属性を一つ以上持つことができるオブジェクトタイプ、postfixTableAliasがあるとして定義しましょう。
また、maildropという属性値は、postfixTableAliasで定義されていることにします。
このエイリアスのノードは、DN:dc=example,dc=uranohoshi,ou=tables,ou=aliases,mail=lilyとなります。

\begin{verbatim}
dn: dc=example,dc=uranohosi,ou=tables,ou=aliases,mail=lily
changetype: add
objectclass: top
objectclass: postfixTableAlias
mail: lily
maildrop: sakurauchi.riko@uranohoshi.example
maildrop: sakurauchi@otonoki.example
\end{verbatim}


\section{属性の追加}

エイリアステーブルのところで説明に使用した、maildropという属性は、オブジェクトタイプinetOrgPersonにはありません。
これは、新しいオブジェクトタイプとして定義したものです。ここでは、スキーマの追加の必要性と、追加方法について説明します。

\subsection{属性追加の必要性}

ISO/ITU-Tによって定義されていない属性を使うときは、スキーマで属性を追加する必要があります。これは、既存の属性では、属性名や属性そのものが足りない場合など、属性を追加します。




\subsection{スキーマファイルによる追加}

\section{OIDとPEN}