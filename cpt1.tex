\chapter{LDAPというデータベース}

\section{ディレクトリサービス}

LDAPは、ディレクトリのサービスです。では、ディレクトリという言葉には、どのような意味があるのでしょうか。研究社の新英和中辞典では、以下のように日本語に訳してあります。

\begin{verbatim}
Directory
(特定の地区などの)住所氏名録、商工人名録
\end{verbatim}

UNIX系のOSでは、ファイルシステム上のパスをディレクトリとよんでいます。これは、いわば、ファイルの置き場所の表し方を、住所に習えたものです。

\subsection{身近なディレクトリ}

ディレクトリサービスは、場所を表すものです。身近なディレクトリサービスとして、住所があります。
例えば、コミケの会場である東京ビッグサイト会議棟の住所は、東京都江東区有明三丁目11番1です。
これをディレクトリの考え方で分解してみると、都道府県が東京都、東京都の中の区が江東区、公徳区の中の町が有明、有明の中の番地が三丁目、三丁目の中の11番、11番の中の1、というようになります。

インターネットでの場所を表すURI(Unique Resoruce Location)も、同様にディレクトリです。たとえば、www.uranohoshi.exampleというホスト名は、exampleというトップレベルドメインを使っているuranohoshiというドメイン名で、ホスト名がwwwという構造です。

先ほど挙げた、ファイルシステム上のパスはどうでしょうか。たとえば、/usr/games/dmという、dm(8)のパスで考えてみましょう。これは、ルートディレクトリの先にあるusr/の、その先にあるgames/の先にあるdm、というように読みます。

\subsection{ディレクトリという考えかた}

ここまでディレクトリの例を取り上げてきました。この、ディレクトリという考え方には特徴があります。ディレクトリは、住所のように、大きな範囲から、部分を取っていき最終的に一つの場所を表すものです。集合論的にいえば、部分空間の部分空間を選択して、最終的にその様相を特定するという考え方です。

では、なぜわざわざディレクトリという考え方をするのでしょうか。住所の例でいけば、東京都江東区有明三丁目11番1ビッグサイト会議棟と書くのではなく、東京都ビッグサイト会議棟、と書いてもいいような気がします。
では、ビッグサイト会議棟という場所をさがすとしたらどうでしょうか。

東京都の中でビッグサイト会議棟という場所を探すとしたら、東京都という範囲をすべて探す必要があります。ですが、東京都江東区の範囲で探せば、豊島区や港区という部分空間は探さないですみます。
更に、東京都江東区有明の範囲で探せば、江東区豊洲や江東区南砂の範囲は探さなくていいことになります。



\subsection{木構造のデータベース}

\section{LDAPツリー}

\subsection{場所をどうあらわすのか}

\subsection{探査の深さと計算量}

木構造のデータベースに